%===============
%一行目に必ず必要
%文章の形式を定義
%===============
\documentclass{ujarticle}
%===============
%パッケージの定義、必要か不明
%===============
%この下4つを加えることで、mathbbが機能した
\usepackage{amsthm}
\usepackage{amsmath}
\usepackage{amssymb}
\usepackage{amsfonts}
%可換図式用パッケージ
\usepackage{amscd}
\usepackage[all]{xy}
\usepackage{tikz-cd}
%リンク用パッケージ
\usepackage[dvipdfmx]{hyperref}
%複数行コメント
%\usepackage{comment}

\usepackage{my-default}
%タイトルデータ
\title{Basic Theory of Representation Theory}
%\author{ari}
\date{\today}


%===============
%定理環境の設定
%セクション毎
%===============


\begin{document}

% タイトルを出力
\maketitle
% 目次の表示
\tableofcontents
\part{1/?}
\section{概要}
初回ということで表現論に至るまでの導入の歴史的な部分と線形代数の基本を説明した.

\begin{thm}
有限次元$K$ベクトル空間$V,W$の間の線形写像全体は$\mathrm{dim}V \times \mathrm{dim}W$の行列全体と一対一対応する
\end{thm}
\part{2/5}
\setcounter{section}{0}
\section{概要}
表現では写像の間の写像が非常に多く出てくるので、写像間の写像等について議論した.

\begin{prob}
$G$から$\mathrm{End}V$への間のモノイド準同型$f$に対し,$f(G)$は群となるか?また、$GL(V)$の部分群となるか?
\end{prob}

\end{document}
